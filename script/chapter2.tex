\chapter{Lorentz Transformation}
\section{Introduction}
Metric (used for distance measuring) is defined as
\begin{align}
   g_{\mu\nu} = \begin{pmatrix} 1 & 0 & 0 & 0 \\ 0 & -1 & 0 & 0 \\ 0 & 0 & -1 & 0 \\ 0 & 0 & 0 & -1\end{pmatrix}.
\end{align}
In string and general relativity, people tend to use the signature $\diag(-, +, +, +)$.

By defining the four-momentum, the relativistic dispersion relation can be given
\begin{align}
   p &= (E, \pmb{p}), \\
   p^2 &= E^2 - \pmb{p}^2 = m^2.
\end{align}

Light is always light-like
\begin{align*}
   k^2 = t^2 - (x^2 + y^2 + z^2) = 0.
\end{align*}

Greek indices always go from $0$ to $3$
\begin{align}
   r^2 = g_{\mu\nu} r^\mu r^\nu = t^2 - \pmb{r}^2.
\end{align}

Distance between two spacetime point is defined via
\begin{align}
   \left| r_A - r_B \right|  = \sqrt{(r_A - r_B)\cdot(r_A - r_B)} = \sqrt{r_A^2 + r_B^2 - 2r_A\cdot r_B}.
\end{align}

\section{Lorentz Transformation}
\textit{Lorentz transformation} is a transformation between two inertial frames moving with constant velocity $\pmb{v}$ with respect to each other (boosts).
\begin{align*}
   x &= (x_0, x_1, x_2, x_3), \\
   x' &= (x'_0, x'_1, x'_2, x'_3), \\
   x_0 &=  t .
\end{align*}

We define (in SI unit)
\begin{align}
   \beta = \frac{v}{c} \quad \text{or} \quad
   \pmb{\beta} = \frac{\pmb{v}}{c}.
\end{align}

Coordinates in these two frames are related like
\begin{align*}
   x'_0 &= \gamma(x_0 - \beta x_1), \\
   x'_1 &= \gamma (x_1 - \beta x_0), \notag \\
   x'_2 &= x_2, \notag \\
   x'_3 &= x_3. \notag \label{math:lorTrafo}
\end{align*}
Inverse transformation can be achieved with $\beta \mapsto -\beta$

$\gamma$-factor is defined via
\begin{align}
   \gamma = \frac{1}{\sqrt{1-{\beta}^2}}.
\end{align}
Since the particles cannot travel faster than light, $|\pmb{\beta}| \leq 1$, then $\gamma \geq 1$.

Alternative parametrization using \underline{rapidity} $\zeta$
\begin{align*}
   \beta &= \tanh(\zeta), \quad \gamma = \cosh(\zeta), \\
   \gamma \beta &= \sinh(\zeta).
\end{align*}

Insert this into equation (\ref{math:lorTrafo})
\begin{align*}
   x'_0 = x_0 \cosh(\zeta) - x_1 \sinh(\zeta), \\
   x'_1 = x_0 \sinh(\zeta) - x_1 \cosh(\zeta).
\end{align*}

We can turn this into matrices
\begin{align}
   \begin{pmatrix} x_0 \\ x_1 \\ x_2 \\ x_3\end{pmatrix}
   &=
   \begin{pmatrix} \gamma & \gamma\beta & 0 & 0 \\ \gamma\beta & \gamma & 0 & 0 \\ 0 & 0 & 1 & 0 \\ 0 & 0 & 0 & 1\end{pmatrix}
   \cdot
   \begin{pmatrix} x'_0\\ x'_1 \\ x'_2 \\ x'_3\end{pmatrix}, \notag \\
   x &= \Lambda x'.
\end{align}

\section{Mathematical Properties of Lorentz Transformation}
Distance is invariant under Lorentz transformation
\begin{align}
   s^2 = x_0^2 - x_1^2 + x_2^2 + x_3^2 = x^2.
\end{align}

Lorentz transformation includes
\begin{itemize}
   \item Rotation and boosts
   \item Parity $\pmb{x} \mapsto -\pmb{x}$
   \item Time reversal $t \mapsto -t$
\end{itemize}
We can also expand it with translation. It then turns to Poincaré group.

\subsection{Tensors}
Define a function of original coordinates ($\alpha=0,1,2,3$)
\begin{align}
   x'^\alpha = x'^\alpha (x^0, x^1, x^2, x^3).
\end{align}

If $x'^\alpha$ transforms like
\begin{align}
   x'^\alpha = \frac{x'^\alpha}{x^\beta} x^\beta,
\end{align}
it is called \underline{contravariant}

Consider derivative $\frac{\partial}{\partial x'^\alpha}$
\begin{align}
   \frac{\partial f(x)}{\partial x'^\alpha} = \frac{\partial f(x)}{\partial x^\beta} \frac{\partial x^\beta}{\partial x'^\alpha}.
\end{align}
We can see the $x'$ is now in the denominator. The objects transformed like this are called \underline{covariant}.

Consider the following generic objects: $A'^\alpha$ contravariant vector
\begin{align}
   A'^\alpha = \frac{\partial x'^\alpha}{\partial x^\beta} A^\beta.
\end{align}

$B'_\alpha$ is covariant
\begin{align}
   B'_\alpha = \frac{\partial x^\beta}{\partial x'^\alpha} B_\beta.
\end{align}
Note $(x^0, x^1, x^2, x^3)$ is contravariant.

The field strength tensor 
\begin{align*}
  F'^{\alpha\beta} = \frac{\partial x'^\alpha}{\partial x^\gamma} \frac{\partial x'^\beta}{\partial x^\delta} F^{\gamma\delta},
\end{align*}
is contravariant rank 2.
Mixed is also allowed 
\begin{align*}
  H'^\alpha_\beta = \frac{\partial x'^\alpha}{\partial x^\gamma} \frac{x^\delta}{\partial x'^\beta} H^\delta_\gamma.
\end{align*}

\subsection{Inner or scalar product}
Lorentz transformation is an isometry with Minkowski scalar product
\begin{align*}
   B' \cdot A' &= B'_\alpha A'^\alpha, \\
               &= \left( \frac{\partial x^\beta}{\partial x'^\alpha} B_\beta \right) \left( \frac{\partial x'^\alpha}{\partial x^\gamma} A^\gamma \right), \\
               &= \frac{\partial x^\beta}{\partial x^\gamma} B_\beta A^\gamma, \\
               &= \delta^\beta_\gamma B_\beta A^\gamma, \\
               &= B \cdot A.
\end{align*}

Infinitesimal distance is
\begin{align}
   \dd{s}^2 &= \left( \dd{x^0} \right)^2 - \left( \dd{\pmb{x}} \right)^2 \notag, \\
            &= (g_{\alpha\beta} \dd{x^\alpha}) \dd{x^\beta} = \dd{x_\beta} \dd{x^{\beta}}.
\end{align}
Thus we can use metric tensor to lower index $\dd{x_\beta} = g_{\alpha\beta} \dd{x^{\alpha}}$

\begin{align*}
   A^\alpha &= \left(A^0, \pmb{A} \right), \\
   A_\alpha &= \left(A^0, -\pmb{A} \right).
\end{align*}

\section{Matrix Representation of Lorentz Transformation}
\subsection{General Properties}
We have 
\begin{align*}
   x = \begin{pmatrix} x^0 \\ x^1 \\ x^2 \\ x^3 \end{pmatrix}, \quad 
   gx =  \begin{pmatrix} x^0 \\ -x^1 \\ -x^2 \\ -x^3 \end{pmatrix}.
\end{align*}
Then $a\cdot b = (a, gb) = g_{\mu\nu} a^\mu b^\nu = (ga, b) = a^Tgb = (ga)^T b$

Lorentz transforming the coordinate
\begin{align}
   x'^\mu = \Lambda^\mu_{\;\nu} x^\nu \Leftrightarrow x' = \Lambda x.
\end{align}

\begin{align*}
   g_{\mu\nu} x^\mu x^\nu &= g_{\sigma \tau} x'^\sigma x'^\tau, \\
                          &= g_{\sigma \tau} \Lambda^\sigma_{\; \mu} x^\mu \Lambda^\tau_{\;\nu} x^\nu, \\
                          &= g_{\sigma \tau} \Lambda^\sigma_{\; \mu} \Lambda^\tau_{\;\nu}x^\mu  x^\nu.
\end{align*}

Then we have the \textit{defining} rule of Lorentz group
\begin{align}
   \begin{split}
    g_{\mu\nu} &= g_{\sigma \tau}  \Lambda^\sigma_{\; \mu} \Lambda^\tau_{\;\nu}, \\
   g &= \Lambda^T g \Lambda.
   \end{split}
\end{align}

\paragraph{Properties} of Lorentz transformation
\begin{itemize}
   \item $\left|\det(\Lambda) \right| = 1$
   \item $\left|\Lambda^0_{\;0} \right| \geq 1$
\end{itemize}

The orthochronous Lorentz transformations ()$\Lambda$ forms a group. \\

Parity does not form a group
      \begin{align}
         \Lambda_P = \text{diag}(1,-1,-1,-1)
      \end{align}

Time reversal 
\begin{align}
   \Lambda_T = \text{diag}(-1,+1,+1,+1)
\end{align}

There are four classes of Lorentz transformations depending on $\left(\sgn(\det(\Lambda)), \sgn(\Lambda^0_0) \right)$
\begin{itemize}
   \item $(+, +)$ $\Lambda$
   \item $(- ,-)$ $\Lambda_T \Lambda$ 
   \item $(-, +)$ $\Lambda_P \Lambda$
   \item $(+, -)$ $\Lambda_T\Lambda_P \Lambda$
\end{itemize}

Orthochronous $\Lambda$ has 6 parameters, $3$ for boosts and $3$ for rotations. $\Lambda^T g \Lambda = g$ is actually $16$ equations. All matrices here are symmetric. Thus $6$ of $16$ are redundant. There are $10$ independent equations.
$\Lambda$ has $16$ entries and it has $16-10=6$ free parameters.

\subsection{Explicit Construction}
We will restrict ourselves in \textit{orthochronous} Lorentz transformations. The exponential function is defines via Taylor expansion. With $L \in \R^{4 \times 4}$
\begin{align*}
   \Lambda = \euler^L = \exp(L).
\end{align*}

From linear algebra we know (because of the properties of determinant and trace)
\begin{align}
   \det(\Lambda) = \det(\euler^L) = \euler^{\tr(L)}.
\end{align}

Since $\det(\Lambda) = 1$, $\tr(L) = 0$,
$gL$ is anti-symmetric
\begin{align*}
   \Lambda^T g \Lambda &= g, \\
   g \Lambda^T g \Lambda &= \id_4, \\
   g \Lambda^T g &= \Lambda^{-1}, \\
   \exp(g L^T g) &= \Lambda^{-1} = \exp(-L), \\
   \Leftrightarrow g L^T g &= -L, \\
   \Leftrightarrow (gL)^T &= - gL.
\end{align*}

Thus
\begin{align*}
   L = \begin{pmatrix} 0 & L_{01} & L_{02} & L_{03} \\ L_{01} & 0 & L_{12} & L_{13} \\ L_{02} & L_{12} & 0 & L_{23} \\ L_{03} & L_{13} & L_{23} & 0\end{pmatrix}.
\end{align*}

Define 6 basis matrices $S_{1,2,3}$ and $K_{1,2,3}$
\begin{align*}
   &S_1 = \begin{pmatrix} 0&0&0&0 \\ 0&0&0&0 \\ 0&0&0&-1 \\ 0&0&1&0\end{pmatrix}, \quad 
   &&S_2 = \begin{pmatrix} 0&0&0&0 \\ 0&0&0&1 \\ 0&0&0&0 \\ 0&-1&0&0\end{pmatrix}, \quad
   &&&S_3 = \begin{pmatrix} 0&0&0&0 \\ 0&0&-1&0 \\ 0&1&0&0 \\ 0&0&0&0\end{pmatrix}, \\
   &K_1 = \begin{pmatrix} 0&1&0&0 \\ 1&0&0&0 \\ 0&0&0&0 \\ 0&0&0&0\end{pmatrix}, \quad 
   &&K_2 = \begin{pmatrix} 0&0&1&0 \\ 0&0&0&0 \\ 1&0&0&0 \\ 0&0&0&0\end{pmatrix}, \quad 
   &&&K_3 = \begin{pmatrix} 0&0&0&1 \\ 0&0&0&0 \\ 0&0&0&0 \\ 1&0&0&0\end{pmatrix}. \quad 
\end{align*}

$S_i$ is the generator of $3$-dimensional rotations and $K_i$ is the generator of $3$-dimensional boosts. \textcolor{red}{WHAT IS THIS?}
\begin{align*}
   \hat{\pmb n} \in \R^3, & \quad |\hat{ \pmb n}| = 1, \\
   \hat{\pmb n} \cdot \pmb{S} &= n_1S_1 + n_2 S_2 + n_3 S_3, \\
   (\hat{\pmb n}\cdot \pmb{S})^3 &= - \hat{\pmb n} \cdot \pmb{S}, \\
   (\hat{\pmb n}\cdot \pmb{K})^3 &= + \hat{\pmb n} \cdot \pmb{S}.
\end{align*}

In the end, with $\pmb{\omega}, \pmb{\zeta} \in \R^3$
\begin{align*}
   L &= -\pmb{\omega} \cdot \pmb{S} - \pmb{\zeta}\cdot \pmb{K}, \\
   \Lambda &= \exp(-\pmb{\omega} \cdot \pmb{S} - \pmb{\zeta}\cdot \pmb{K}).
\end{align*}

$\pmb{\omega}$ is the axis of rotation, $|\pmb{\omega}|$ is then the angle of rotation.
$\tanh{|\pmb{\zeta}|} = \beta$ and $\frac{\pmb{\zeta}}{|\pmb{\zeta}|}$ is the direction of boost.

We now will look at concrete examples
\begin{itemize}
   \item $\pmb{\zeta} = 0,\; \pmb{\omega} = w \hat{e}_z$
      \begin{align*}
         \Lambda = \begin{pmatrix} 1 & 0 & 0 & 0 \\ 0 & \cos{\omega} & \sin{\omega} & 0 \\ 0 & -\sin{\omega} & \cos{\omega} & 0 \\ 0 & 0 & 0 & 0\end{pmatrix}.
      \end{align*}
      Rotational angle is $\omega$.
   \item $\pmb{\omega} = 0, \; \pmb{\zeta} = \zeta \hat{e}_x$
      \begin{align*}
         \Lambda = \begin{pmatrix} \cosh{\zeta} & -\sinh{\zeta} & 0 & 0 \\ -\sinh{\zeta} & \cosh{\zeta} & 0 & 0 \\ 0 & 0 & 1 & 0 \\ 0 & 0 & 0 & 1\end{pmatrix}.
      \end{align*}
\end{itemize}

Pure general boost $\pmb{\zeta}$
\begin{align*}
   \Lambda &= \exp(-\pmb{\zeta} \cdot \pmb{K}), \\
   \pmb{\zeta} &= \frac{\pmb{\beta}}{|\pmb{\beta}|} \tanh^{-1} |\pmb{\beta}|, \quad \hat{\pmb \beta} = \frac{\pmb{\beta}}{|\pmb{\beta}|}, \\
   \Lambda &= \exp(-\hat{\pmb \beta} \cdot \pmb{K} \tanh^{-1}(\beta)).
\end{align*}

\subsection{Algebra of generators}
Consider the commutation algebra of $S_{i=1,2,3}$ and $K_{i=1,2,3}$
\begin{align}
   \left[ S_i, S_j \right] &= \epsilon_{ijk} S_k,  \\
   \left[ S_i, K_j \right] &= \epsilon_{ijk} K_k, \\
   \left[ K_i, K_j \right] &= - \epsilon_{ijk} S_k.
\end{align}
The last equation causes Thomas precession in atomic physics.

Choose a different basis
\begin{align}
   \pmb{S}_+ = \frac{1}{2} \left( \pmb{S} + i \pmb{K} \right), \\
   \pmb{S}_- = \frac{1}{2} \left( \pmb{S} - i \pmb{K} \right).
\end{align}
Then we can calculate the algebra
\begin{align}
   \left[ S_{+, i}, S_{+, j} \right] &= i \epsilon_{ijk} S_{+, k}, \\
   \left[ S_{-, i}, S_{-, j} \right] &= i \epsilon_{ijk} S_{-, k}, \\
   \left[ S_{+, i}, S_{-, j} \right] &= 0.
\end{align}
In other word, the algebras are decoupled. This familiar algebra is angular momentum algebra $\SU(2)$.

Classification by two numbers ($j_+$, $j_-$)
\begin{align*}
   j_+ = 0, \frac{1}{2}, 1, \dots, \\
   j_- = 0, \frac{1}{2}, 1, \dots. \\
\end{align*}
Dimension $=(2j_+ + 1) (2j_- + 1)$. 

A field is scalar field if $j_+ = j_- = 0$. One fundamental example of scalar field is Higgs boson. Other scalar particles are just bound states.

There are two possible states with spin $\frac{1}{2}$: $(j_+, j_-) = (\frac{1}{2}, 0)$ and $(0, \frac{1}{2})$. For  $(\frac{1}{2}, 0)$ it is
\begin{align*}
   \pmb{S}_+ &= \frac{1}{2} \pmb{\sigma}, \\
   \pmb{S}_- &= 0, \\
   \pmb{S} &= \frac{1}{2} \pmb{\sigma}, \\
   i\pmb{K} &=  \frac{1}{2} \pmb{\sigma} .
\end{align*}
For $(0, \frac{1}{2})$ there is "-".

So two types of spin $\frac{1}{2}$ fermions
\begin{align*}
   \left(\frac{1}{2}, 0\right) \rightarrow e^-_L, \\
   \left(0, \frac{1}{2}\right) \rightarrow e^-_R. \\
\end{align*}
They have different transformation law. W boson only to $e^-_L$ but photon couples to both. 

Under parity transformation $e^-_L \leftrightarrow e^-_R$. Both particles are needed for theory to be invariant under parity transformation, like EM and strong interactions.
